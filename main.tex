%%
%% This is file `sample-acmsmall-conf.tex',
%% generated with the docstrip utility.
%%
%% The original source files were:
%%
%% samples.dtx  (with options: `all,proceedings,bibtex,acmsmall-conf')
%% 
%% IMPORTANT NOTICE:
%% 
%% For the copyright see the source file.
%% 
%% Any modified versions of this file must be renamed
%% with new filenames distinct from sample-acmsmall-conf.tex.
%% 
%% For distribution of the original source see the terms
%% for copying and modification in the file samples.dtx.
%% 
%% This generated file may be distributed as long as the
%% original source files, as listed above, are part of the
%% same distribution. (The sources need not necessarily be
%% in the same archive or directory.)
%%
%%
%% Commands for TeXCount
%TC:macro \cite [option:text,text]
%TC:macro \citep [option:text,text]
%TC:macro \citet [option:text,text]
%TC:envir table 0 1
%TC:envir table* 0 1
%TC:envir tabular [ignore] word
%TC:envir displaymath 0 word
%TC:envir math 0 word
%TC:envir comment 0 0
%%
%% The first command in your LaTeX source must be the \documentclass
%% command.
%%
%% For submission and review of your manuscript please change the
%% command to \documentclass[manuscript, screen, review]{acmart}.
%%
%% When submitting camera ready or to TAPS, please change the command
%% to \documentclass[sigconf]{acmart} or whichever template is required
%% for your publication.
%%
%%
\documentclass[acmsmall,screen,review,anonymous]{acmart}
%%
%% \BibTeX command to typeset BibTeX logo in the docs
\AtBeginDocument{%
  \providecommand\BibTeX{{%
    Bib\TeX}}}

%% Rights management information.  This information is sent to you
%% when you complete the rights form.  These commands have SAMPLE
%% values in them; it is your responsibility as an author to replace
%% the commands and values with those provided to you when you
%% complete the rights form.
\setcopyright{acmlicensed}
\copyrightyear{2018}
\acmYear{2018}
\acmDOI{XXXXXXX.XXXXXXX}
%% These commands are for a PROCEEDINGS abstract or paper.
\acmConference[Conference acronym 'XX]{Make sure to enter the correct
  conference title from your rights confirmation email}{June 03--05,
  2018}{Woodstock, NY}
%%
%%  Uncomment \acmBooktitle if the title of the proceedings is different
%%  from ``Proceedings of ...''!
%%
%%\acmBooktitle{Woodstock '18: ACM Symposium on Neural Gaze Detection,
%%  June 03--05, 2018, Woodstock, NY}
\acmISBN{978-1-4503-XXXX-X/2018/06}


%%
%% Submission ID.
%% Use this when submitting an article to a sponsored event. You'll
%% receive a unique submission ID from the organizers
%% of the event, and this ID should be used as the parameter to this command.
%%\acmSubmissionID{123-A56-BU3}

%%
%% For managing citations, it is recommended to use bibliography
%% files in BibTeX format.
%%
%% You can then either use BibTeX with the ACM-Reference-Format style,
%% or BibLaTeX with the acmnumeric or acmauthoryear sytles, that include
%% support for advanced citation of software artefact from the
%% biblatex-software package, also separately available on CTAN.
%%
%% Look at the sample-*-biblatex.tex files for templates showcasing
%% the biblatex styles.
%%

%%
%% The majority of ACM publications use numbered citations and
%% references.  The command \citestyle{authoryear} switches to the
%% "author year" style.
%%
%% If you are preparing content for an event
%% sponsored by ACM SIGGRAPH, you must use the "author year" style of
%% citations and references.
%% Uncommenting
%% the next command will enable that style.
%%\citestyle{acmauthoryear}



\usepackage{listings}
\usepackage{xcolor}
\usepackage{xspace}
\usepackage{algorithm}
\usepackage{algorithmic}
\usepackage{pifont}
\usepackage{enumitem}
\usepackage{calc}
\usepackage{mdframed}
\usepackage{subcaption}

\newcommand{\lighttechname}{AgenticNorm\xspace}
\newcommand{\trainticket}{TrainTicket\xspace}
\newcommand{\nicefish}{NiceFish\xspace}

\lstdefinelanguage{json}{
    basicstyle=\ttfamily\small,
    numbers=left,
    numberstyle=\tiny,
    stepnumber=1,
    numbersep=5pt,
    showstringspaces=false,
    breaklines=true,
    frame=single,
    morestring=[b]",
    morecomment=[l]{//},
    stringstyle=\color{blue},
    keywordstyle=\color{teal}\bfseries,
}


%%
%% end of the preamble, start of the body of the document source.
\begin{document}


%%
%% The "title" command has an optional parameter,
%% allowing the author to define a "short title" to be used in page headers.
\title{\lighttechname: Multi-Agent Lightweight Anomaly Detection for Web Applications}

%%
%% The "author" command and its associated commands are used to define
%% the authors and their affiliations.
%% Of note is the shared affiliation of the first two authors, and the
%% "authornote" and "authornotemark" commands
%% used to denote shared contribution to the research.

\author{Wenjie Zhang}
\email{wjzhang@nus.edu.sg}
\orcid{0000-0002-2669-1837}
\affiliation{%
  \institution{National University of Singapore}
  \country{Singapore}
}

\author{Yun Lin}
\affiliation{%
  \institution{Shanghai Jiao Tong University}
  \city{Shanghai}
  \country{China}
}
\email{lin_yun@sjtu.edu.cn}
\authornote{Corresponding author}

\author{Yeap Rayson}
\email{rayson.yeap@u.nus.edu}
\affiliation{%
  \institution{National University of Singapore}
  \country{Singapore}
}

\author{Ruikun Pu}
\email{prk4love@sjtu.edu.cn}
\affiliation{%
  \institution{Shanghai Jiao Tong University}
  \city{Shanghai}
  \country{China}
}

\author{Kwok Chun Fung Amos}
\email{e1373883@u.nus.edu}
\affiliation{%
  \institution{National University of Singapore}
  \country{Singapore}
}

\author{Xiwen Teoh}
\email{xiwen.teoh@u.nus.edu}
\affiliation{%
  \institution{National University of Singapore}
  \country{Singapore}
}

\author{Lu Kuai}
\email{kuailu.sh@chinatelecom.cn}
\affiliation{%
  \institution{China Telecom Corporation Limited Shanghai Branch}
  \city{Shanghai}
  \country{China}
}

\author{Jin Song Dong}
\email{dcsdjs@nus.edu.sg}
\affiliation{%
  \institution{National University of Singapore}
  \country{Singapore}
}

% %%
% %% By default, the full list of authors will be used in the page
% %% headers. Often, this list is too long, and will overlap
% %% other information printed in the page headers. This command allows
% %% the author to define a more concise list
% %% of authors' names for this purpose.
\renewcommand{\shortauthors}{Zhang et al.}

\begin{abstract}
Web applications support important areas such as finance, e-commerce, and healthcare, making their reliability and security of paramount importance. 
However, web frontends are inherently manipulable, and abnormal client behaviors may evade backend checks, creating exploitable vulnerabilities. 
Log analysis has emerged as an effective defense by capturing client-server interactions.
Existing approaches typically learn models from logs, either by training neural network models for anomaly classification or by directly using LLMs for anomaly judgment.
However, these model-learning-based approaches suffer from poor interpretability and high false positive rates, making practical deployment challenging.
To address these limitations, our prior work WebNorm proposed a rule-learning-based approach that extracts explicit constraints from logs, providing stronger interpretability.
However, WebNorm relies heavily on program instrumentation for source code analysis, heavyweight proprietary language models, and manually engineered prompts, limiting its broader applicability.

This paper presents \lighttechname, a significant extension of WebNorm that addresses these limitations through a lightweight anomaly detection framework built upon locally deployable LLMs. 
Building on WebNorm's foundation of constraint-based anomaly detection, we extend the approach with three key innovations: 
(1) eliminating source-code dependence via frequency-based inter-API relation discovery, enabling application to closed-source systems, 
(2) reducing log complexity through field clustering into semantically coherent groups, making lightweight models feasible for large-scale logs, and 
(3) iteratively refining prompts with adversarially generated attack logs, overcoming prompt sensitivity and manual engineering requirements. 
These components are integrated into a novel multi-agent workflow that progressively improves anomaly detection without extensive human intervention.  

We conduct comprehensive experiments on popular benchmarks, including TrainTicket and NiceFish, covering over 220k log entries and 230 attack cases.
Results demonstrate that \lighttechname significantly outperforms the original WebNorm, achieving F1-scores of 0.92 and 0.85 compared to WebNorm's 0.88 and 0.75, while requiring significantly less contextual information and eliminating the need for source code access and proprietary models.
\end{abstract}

%%
%% The code below is generated by the tool at http://dl.acm.org/ccs.cfm.
%% Please copy and paste the code instead of the example below.
%%
\begin{CCSXML}
<ccs2012>
 <concept>
  <concept_id>00000000.0000000.0000000</concept_id>
  <concept_desc>Do Not Use This Code, Generate the Correct Terms for Your Paper</concept_desc>
  <concept_significance>500</concept_significance>
 </concept>
 <concept>
  <concept_id>00000000.00000000.00000000</concept_id>
  <concept_desc>Do Not Use This Code, Generate the Correct Terms for Your Paper</concept_desc>
  <concept_significance>300</concept_significance>
 </concept>
 <concept>
  <concept_id>00000000.00000000.00000000</concept_id>
  <concept_desc>Do Not Use This Code, Generate the Correct Terms for Your Paper</concept_desc>
  <concept_significance>100</concept_significance>
 </concept>
 <concept>
  <concept_id>00000000.00000000.00000000</concept_id>
  <concept_desc>Do Not Use This Code, Generate the Correct Terms for Your Paper</concept_desc>
  <concept_significance>100</concept_significance>
 </concept>
</ccs2012>
\end{CCSXML}

\ccsdesc[500]{Do Not Use This Code~Generate the Correct Terms for Your Paper}
\ccsdesc[300]{Do Not Use This Code~Generate the Correct Terms for Your Paper}
\ccsdesc{Do Not Use This Code~Generate the Correct Terms for Your Paper}
\ccsdesc[100]{Do Not Use This Code~Generate the Correct Terms for Your Paper}

%%
%% Keywords. The author(s) should pick words that accurately describe
%% the work being presented. Separate the keywords with commas.
\keywords{Do, Not, Use, This, Code, Put, the, Correct, Terms, for,
  Your, Paper}
%% A "teaser" image appears between the author and affiliation
%% information and the body of the document, and typically spans the
%% page.
% \begin{teaserfigure}
%   \includegraphics[width=\textwidth]{sampleteaser}
%   \caption{Seattle Mariners at Spring Training, 2010.}
%   \Description{Enjoying the baseball game from the third-base
%   seats. Ichiro Suzuki preparing to bat.}
%   \label{fig:teaser}
% \end{teaserfigure}

\received{20 February 2007}
\received[revised]{12 March 2009}
\received[accepted]{5 June 2009}



%%
%% This command processes the author and affiliation and title
%% information and builds the first part of the formatted document.
\maketitle


{\color{red}{TODOS: 1) replace all "attack" to abnormal. 2) replace invaraints to constraints 3) place novelty to adversal learning. 4) add more examples. 5) add more details in method}}

\section{Introduction}

Web applications play a critical role in modern infrastructures, supporting domains such as finance~\cite{feyen2021fintech,vukovic2025ai}, e-commerce~\cite{rahman2022revolutionizing}, and healthcare~\cite{lazakidou2009web}. Their reliability and security are of paramount importance. Unfortunately, web frontends are inherently manipulable: attackers can alter client-side code or parameters to bypass validations, tamper with workflows, or inject abnormal behaviors.
通常情况下,web application的后端会在代码中加入权限验证等措施来防止这种对于前端的篡改,然而由于这些异常行为难以被通常的前端测试所覆盖,因此可能存在一些异常行为没有完全被后端应用程序所检测,从而导致漏洞的发生。

为了防止潜在的攻击行为,日志分析成为了一种有效的手段。在Web系统中,日志通常记录了客户端和服务端之间所有的交互细节,包括API调用、请求参数、响应状态等信息。通过分析这些日志,可以识别出异常的交互模式,从而发现潜在的安全威胁\cite{yen2013beehive,alam2019framework}。目前,state-of-the-art的日志分析方法主要分为两类:(1) model learning approaches that learns a predictive model from normal logs and use it to identify anomalies~\cite{acharya2007mining,lorenzoli2008automatic,walkinshaw2008inferring,pradel2009automatic,beschastnikh2011leveraging,krka2014automatic,breier2015anomaly,amar2018using,rufino2020improving,stocco2020towards,kang2019spatiotemporal,njoku2025kernel,wu2023effectiveness,lupton2021literature,alam2019framework,schneider2010synoptic}, and (2) rule learning approaches that mine logical constraints from normal logs and detect violations against them~\cite{liao2024detecting}.

第一类model learning based approaches 主要通过一个深度学习模型学习正常与异常日志的行为,对于日志给出二分类\cite{du2017deeplog,brown2018recurrent},然而这种方式往往缺乏可解释性,模型的输出结果仅为二分类结果,无法给出具体的错误原因,由于日志数量庞大,即使只有很小的false positive发生率,导致的误报也非常多,给实践部署进一步带来论难,且这种方法对于一些细微的异常行为难以捕捉\cite{no2024training},如果攻击者的篡改行为与正常行为非常相似,模型往往难以区分,从而导致漏报。

为了解决第一类方法中的弊端,第二类rule learning based approaches 主要通过挖掘日志中的逻辑约束来进行异常检测\cite{liao2024detecting},这种方法往往具有较好的可解释性,能够给出具体的错误原因,然而现有的方法仍有下面几个弊端:
It detects abnormalities like the \trainticket case by learning cross-API constraints (e.g., \texttt{cancelOrder.arguments.orderId} must appear in \texttt{queryOrders.results[].id}).
While effective, \textsc{WebNorm} has three limitations:

\begin{itemize}
    \item \textbf{Program-analysis and source-code dependence}: it requires access to, and instrumentation of, frontend/backend code to align logs with code-level flows, which is costly, brittle under rapid changes, and infeasible for closed-source or third-party components.
    \item \textbf{Heavyweight and proprietary LLMs}: constraint confirmation/synthesis relies on large, closed-source proprietary models, introducing latency, cost, and compliance/privacy concerns. Moreover, real logs are long and deeply nested, often exceeding the context window of compact models, further pushing deployments toward heavyweight remote models.
    \item \textbf{Prompt sensitivity}: the correct constraints often emerge only when the prompt is carefully crafted, leading to manual, project-specific prompt engineering with limited transferability and unstable results.
\end{itemize}

In this paper, we propose \lighttechname, a lightweight anomaly detection framework designed around compact, locally deployable LLMs. 
Unlike WebNorm, \lighttechname avoids program-analysis dependence, removes the need for heavyweight proprietary models, and mitigates prompt fragility through two key techniques:

\begin{itemize}
    \item \textbf{Eliminating source-code dependence.} Instead of relying on program instrumentation to build log-code mappings, \lighttechname directly learns constraints from raw logs. It discovers inter-API relationships using frequency-based analysis of co-occurring calls, and derives constraints purely from runtime behaviors, making the framework applicable even when source code is unavailable.
    \item \textbf{Field Clustering for Context Reduction.} To overcome the limitation of compact models on long and nested logs, \lighttechname expands JSON entities into flattened fields and groups them into semantically coherent clusters. Each cluster is processed independently, greatly reducing the context length while preserving meaningful comparisons. This enables lightweight models to handle large-scale logs without resorting to heavyweight remote LLMs.
    \item \textbf{Prompt Refinement via Generated Abnormals.} To address the fragility of manual prompt engineering, \lighttechname introduces an iterative loop where adversarial logs are automatically generated to expose missing constraints. These logs are then used to refine prompts, producing project-specific instructions that are stable and transferable across iterations. This process allows compact models to progressively capture more accurate and robust constraints without human-crafted prompts.
\end{itemize}

\lighttechname integrates these components into a multi-agent workflow, where agents for invariant generation, attack generation, and prompt refinement cooperate iteratively. The result is a system that adapts and self-improves without extensive human involvement.

\paragraph{Contributions.} This paper makes the following contributions:
\begin{itemize}
    \item We propose \lighttechname, a multi-agent framework for web anomaly detection that eliminates the dependency on application source code.
    \item We implement the \lighttechname framework and evaluate it on real-world benchmarks, including TrainTicket and NiceFish.  
    \item Experimental results demonstrate that \lighttechname requires less contextual information while achieving more effective anomaly detection compared to existing approaches.
\end{itemize}

\section{Motivating Example}

\begin{figure}
    \centering
    \includegraphics[width=\textwidth]{figures/motivating-example.pdf}
    \caption{A motivating example from the \trainticket dataset illustrating a ticket refund workflow, 
where the user first retrieves a list of orders using \texttt{queryOrders} and then requests a refund through \texttt{cancelOrder}. 
The blue highlights indicate identifiers that should match across the two APIs, while the red highlight marks an attack case, 
where the \texttt{orderId} is replaced with a value not present in the queried list.}
    \label{fig:motivating_example}
\end{figure}


To illustrate the challenges of detecting anomalies from API logs, we examine a case from the \trainticket dataset~\cite{trainticketsystem} involving a compromised ticket refund process. 
We focus on two backend APIs that appear in the logs: 
\texttt{/api/v1/queryOrders} (\texttt{queryOrders}) and \texttt{/api/v1/cancelOrder} (\texttt{cancelOrder}). 
Figure~\ref{fig:motivating_example} shows a simplified version of the relevant log entries, while the complete logs are provided in our anonymous artifact~\cite{agenticnorm-website}. The original logs are lengthy and represented in JSON format.  

In a normal cancellation workflow, the operation consists of two steps. 
First, the user invokes \texttt{queryOrders} to retrieve a list of refundable tickets. 
Second, the user selects one of the returned tickets and issues a cancellation request via \texttt{cancelOrder}. 
In this case, the \texttt{orderId} used for cancellation (e.g., \texttt{fe9c72d9}) must be one of the identifiers returned by \texttt{queryOrders}.  

However, because frontend code executes entirely on the client side, a malicious user can tamper with browser data and craft an unauthorized request. 
For instance, the attacker may replace the legitimate \texttt{orderId} with an arbitrary identifier not included in the queried list (e.g., \texttt{418ea03c}). 
This manipulation allows the attacker to cancel a ticket they do not own or repeat a cancellation that should be disallowed. 
Such behavior may lead to duplicate refunds and financial losses, creating serious risks for system security and integrity.  


\subsection{Background}
Here, we briefly introduce the concepts of APIs and logs in web applications, and illustrate them with the example in Figure~\ref{fig:motivating_example}.  
An API serves as the communication interface between the frontend and backend, typically specified by a request path and a payload format. 
 In this example, there are two APIs. The first API, \texttt{queryOrders}, returns a list of refundable tickets, each with a unique \texttt{orderId}. The second API, \texttt{cancelOrder}, takes an \texttt{orderId} as input to process a cancellation request.

A log records the actual data exchanged through an API call, usually including the request path and a JSON-structured object for both request and response.
Thus, logs are often stored in structured JSON format.  
In the given example, for each API, we show its corresponding log entry in a simplified format.

For anomaly detection, we consider two types of logs: normal and abnormal.
Logs generated by normal browser operations are considered benign, while those modified or replayed on the client side are treated as abnormal.

In a normal case, the second API, \texttt{cancelOrder}, should use one of these \texttt{orderId}s when submitting a cancellation request (e.g., \texttt{fe9c72d9}), this invariant is enforced by the frontend code.
However, an attacker may tamper with the frontend and replace this value with an arbitrary \texttt{orderId} not included in the list (e.g., \texttt{418ea03c}), thereby cancelling a ticket they do not own.  

The relation that \texttt{cancelOrder.arguments.orderId} must match one of \texttt{queryOrders.results[].id} is an example of a cross-API dependency. 
WebNorm refers to such relations as \emph{dependencies} and identifies them through program analysis, source code instrumentation, or other static/dynamic methods. 
Once the dependency is known, any violation of it in the logs can be flagged as abnormal behavior.

\subsection{Existing Approaches}  
Existing log-based anomaly detection methods can be divided into two categories:  
(1) \textbf{Model-learning-based detectors}, which learn embeddings or features from normal and abnormal logs and classify anomalies based on learned patterns; and  
(2) \textbf{Invariant-learning-based detectors}, which derive semantic invariants from logs and flag violations as anomalies.  

Model-learning-based detectors face difficulties in this scenario. Normal and abnormal logs differ in only a few fields, making them nearly indistinguishable in embedding space. 
Moreover, the anomaly signal is diluted by long sequences interleaved with irrelevant events, further reducing detection accuracy.  

To address these issues, WebNorm~\cite{liao2024detecting} learns semantic invariants from logs using LLMs. 
For example, it can capture the constraint that \texttt{cancelOrder.arguments.orderId} must match one of \texttt{queryOrders.results[].id}. 
WebNorm maps code-level data flows to log fields and asks an LLM to validate them as invariants. At runtime, any violation of such constraints is flagged as an anomaly.  
In this motivating case, WebNorm successfully identifies the cross-API constraint and detects the abnormal behavior.  

Despite its effectiveness, WebNorm has three key limitations:  
\begin{itemize}
    \item \textbf{Dependence on program analysis and source code}: it requires access to and instrumentation of frontend/backend code to align logs with program workflows, which is costly, fragile under rapid iteration, and infeasible for closed-source or third-party systems.  
    \item \textbf{Reliance on heavyweight proprietary LLMs}: invariant synthesis depends on closed-source remote models, leading to latency, cost, and compliance/privacy concerns. Real logs are long and deeply nested, often exceeding the context window of compact models.  
    \item \textbf{Prompt sensitivity}: correct invariants often emerge only when carefully engineered prompts are used, making the process labor-intensive, project-specific, and difficult to generalize.  
\end{itemize}  

\subsection{Our Approach}  
We aim to preserve WebNorm’s strength in capturing \emph{consistency constraints} while addressing its limitations. 
Our approach uses only logs (no program analysis), relies primarily on compact, locally deployable LLMs, and mitigates prompt fragility by automatically refining prompts with generated abnormal logs.  

Specifically, we propose two techniques for anomaly detection with compact models:  
(1) \emph{Field Clustering}: log entities are expanded into flattened fields and grouped into small, semantically coherent clusters. Each cluster is processed separately by the LLM, reducing context length and highlighting field-level relationships; and  
(2) \emph{Prompt Refinement via Generated Attacks}: prompts are designed to guide the LLM to generate abnormal logs. These logs reveal missing constraints, which are then used to refine the prompts. The refined prompts enable compact models to generate more accurate invariants.  

\paragraph{Field Clustering}  
To handle compact models efficiently, we expand each JSON log into flattened fields and group comparable fields into clusters. 
Figure~\ref{fig:motivating_example_clustering} illustrates this process. Each cluster is provided as a separate input to the LLM, ensuring that semantically related fields are explicitly compared.  

\begin{figure}
    \centering
    \includegraphics[width=\textwidth]{figures/motivating-example-clustering.pdf}
    \caption{An illustration of \emph{Field Clustering} on the motivating example from the \trainticket dataset. The original log entity is decomposed into flatten fields and then clustered into clusters, each containing only related and comparable fields.}
    \label{fig:motivating_example_clustering}
\end{figure}
  

\paragraph{Prompt Refinement via Generated Attacks}  

\begin{figure}
    \centering
    \includegraphics[width=\textwidth]{figures/motivating-example-prompt.pdf}
    \caption{An illustration of different prompts. The left panel shows the original prompt used in WebNorm, which relies on manually crafted instructions and examples, making prompt adjustment highly labor-intensive. The middle panel adds new instructions to the original prompt, guiding the LLM to generate abnormal logs. The right panel presents the refined prompt, which is able to capture a broader set of invariants.}
    \label{fig:motivating_example_prompt}
\end{figure}
  

Compact models often lack reasoning power and cannot reliably infer constraints from fixed prompts. 
Manually adjusting prompts is both labor-intensive and project-specific. 
Figure~\ref{fig:motivating_example_prompt} (left) shows the original WebNorm prompt, which depends on handcrafted instructions. 
Our approach (middle) enhances the prompt by adding instructions that guide the model to generate abnormal logs. These generated logs help the model reason about field relationships—for example, ensuring consistency between joined fields and their originals. 
As shown in Figure~\ref{fig:motivating_example_prompt} (right), the refined prompt enables the model to produce invariants that successfully capture the intended field-level constraints.

\section{Method}


\begin{figure}
    \centering
    \includegraphics[width=.7\textwidth]{figures/method-overview.pdf}
    \caption{Method Overview}
    \label{fig:method_overview}
\end{figure}


In this section, we present \lighttechname, a lightweight multi-agent anomaly detection framework for web applications.
\lighttechname is designed to overcome the limitations of prior solutions such as WebNorm, namely the reliance on heavyweight proprietary models, sensitivity to prompt engineering, and difficulty in handling long log contexts.
Figure~\ref{fig:method_overview} provides an overview of the workflow.

\subsection{Workflow}

A central challenge in log-based anomaly detection is that prompt quality strongly influences detection results. Fixed prompts are brittle and may fail to capture certain constraints, leading to missed anomalies. To address this limitation, we propose an iterative loop in which attack generation and prompt adjustment are tightly coupled. The loop continuously strengthens prompts by exposing them to adversarial scenarios that exploit their current weaknesses. This process consists of three main modules, forming an iterative loop (Figure~\ref{fig:method_overview}):

\lighttechname consists of three main modules:
\begin{itemize}
    \item \textbf{Constraint Learning}: derives constraints from normal logs.
    \item \textbf{Attack Generation}: synthesizes attack logs that break or bypass the learned constraints.
    \item \textbf{Prompt Refinement}: updates LLM prompts using feedback from undetected attacks.
\end{itemize}

\lighttechname begins by deriving constraints from normal logs using an initial prompt in the \textbf{Constraint Learning} module.
Then, the \textbf{Attack Generation} module synthesizes attack logs that break or bypass the learned constraints.
Finally, the \textbf{Prompt Refinement} module updates LLM prompts using feedback from undetected attacks.

This adversarial loop allows prompts to evolve dynamically. Each cycle expands the attack space by introducing logs that specifically target the weaknesses of the current constraints, and in turn strengthens the prompts by incorporating counterexamples. Over time, this reduces reliance on manual intervention and improves robustness against both known and novel attacks.

Next, we break down each module in detail.

\subsection{Constraint Learning}


\begin{figure}
    \centering
    \includegraphics[width=\textwidth]{figures/method-constraint-learning.pdf}
    \caption{Constraint Learning}
    \label{fig:method_constraint}
\end{figure}


\lighttechname generally follows the idea of WebNorm, but differs in that it does not rely on source code or data-flow analysis.
This requires us to replace several of its original components.
Figure~\ref{fig:method_constraint} illustrates the process of constraint learning.
First, \lighttechname discovers relationships between APIs through frequency-based analysis.
Next, to adapt to lightweight LLMs, \lighttechname applies \emph{Field Clustering}, which reduces the length of the input context per query, thereby lowering the workload of the model while improving its ability to identify constraints.
Finally, \lighttechname adopts a similar approach to WebNorm for detecting both intra-API and inter-API constraints, using an LLM to extract constraints and generate corresponding Python checking code.
Unlike WebNorm, however, the prompts employed here are not manually designed; instead, they are obtained from the iterative refinement process described later, making them better suited for lightweight LLMs.

\subsubsection{Frequency Analysis}

\lighttechname employs a frequency-based method to identify related APIs, eliminating the need for program analysis.
Specifically, for a given API, it scans the surrounding window of log entries and counts the frequency of co-occurring API calls.
The top-$K$ most frequent co-occurrences are considered related APIs, thus establishing inter-API relations.
After this step, we utilize an LLM to verify and filter out spurious relations.

\begin{figure}
    \centering
    \includegraphics[width=\textwidth]{figures/method-freq.pdf}
    \caption{An example of frequency-based analysis.}
    \label{fig:method-freq}
\end{figure}


Figure~\ref{fig:method-freq} shows an example of frequency-based analysis on the \trainticket dataset.
Given a list of API calls, we slide a window of size $K$ and count the frequency of co-occurring APIs.
For instance, in this case, \texttt{CreateOrder} and \texttt{AddPassenger} frequently appear together, indicating a potential relationship.
Then, the LLM is used to verify and filter out spurious relations.

\subsubsection{Field Clustering}

Lightweight LLMs are constrained by limited context windows, making it infeasible to directly process lengthy and complex logs. To address this limitation, we introduce \emph{field clustering}, a technique that decomposes log entries into semantically related groups. This allows constraints to be extracted while ensuring that the input remains within the restricted context length.

To this end, \lighttechname first analyzes the structure of logs, which often contain nested dictionaries and arrays. It then applies a set of expansion rules to flatten these structures into atomic fields. Finally, it employs an LLM to cluster the expanded fields based on semantic relatedness, forming lightweight groups that can be processed within the context limits.

Figure~\ref{fig:motivating_example_clustering} provides an example of the field clustering process. The original log contains nested dicts (e.g., \texttt{arguments}, \texttt{qi}, etc.) and arrays (e.g., \texttt{results}). These structures are first expanded into flat fields (e.g., \texttt{arguments.loginId}, \texttt{arguments.orderId}, etc.). Finally, the expanded fields are clustered into semantically related groups (e.g., the cluster of \texttt{arguments.loginId}, \texttt{env.userId}, \texttt{qi.accountId} represents user identifiers).

Formally, we show the field clustering process in three steps: Log Structure Discovery, Expansion, and Clustering.
Log Structure Discovery identifies the schema of logs and their data types.
Expansion applies a set of rules to flatten nested structures into atomic fields.
Clustering groups the expanded fields into semantically related clusters using an LLM.



\begin{figure}
    \centering
    \myfigfontsize
    \begin{mdframed}
        \[
            \begin{aligned}
                \text{Data} :=\; & \texttt{unknown}                                            \\
                \;|\;            & \texttt{bool}                                               \\
                \;|\;            & \texttt{number}                                             \\
                \;|\;            & \texttt{string}                                             \\
                \;|\;            & \texttt{dict}[key_1:\text{Data}, key_2:\text{Data}, \ldots] \\
                \;|\;            & \texttt{array}[\text{Data}]
            \end{aligned}
        \]
    \end{mdframed}
    \caption{Grammar for Log Data Types}
    \label{fig:log-data-grammar}
\end{figure}

\paragraph{Log Structure Discovery.}
Each API may produce logs with diverse structures, including nested dictionaries and arrays. We first parse the logs to uncover their structural schema and data types. For each API, \lighttechname analyzes all log entries and infers a unified schema that captures the common structure, represented as fields and their associated types. Formally, we define a recursive grammar for data types shown in Figure~\ref{fig:log-data-grammar}.

Here, \texttt{unknown} denotes cases where the log structure cannot be precisely determined (e.g., due to ambiguity or inconsistency). By aggregating logs across APIs, \lighttechname derives unified schemas that reconcile structural variations.

\paragraph{Expansion.}
After schema discovery, we apply a set of expansion rules to transform nested structures into flat fields. This ensures that all relevant information is explicitly represented, thereby facilitating clustering and constraint generation.

The rules are as follows:
\begin{itemize}
    \item \textbf{Dict Expansion}: For a dictionary value, each key is concatenated with its parent field using a dot ``.'' separator. Formally, \texttt{d: dict[key: value]} is expanded into \texttt{"d.key": value}.
    \item \textbf{Array of Dict Expansion}: For an array of dictionaries, each dictionary key is expanded to a new array field. Formally, \texttt{a: array[dict[key: value]]} is expanded into \texttt{"a[].key": array[value]}.
    \item \textbf{Field Joining}: In certain cases, meaningful semantics emerge when fields from different structural levels are \emph{joined}. Specifically, if an outer field and an inner field share a common key (e.g., an identifier), we match the entry and promote it as a new joined field.
\end{itemize}

Here are some examples of the expansion rules in motivating example:
\begin{itemize}
    \item \textbf{Dict Expansion}: \texttt{arguments: dict[loginId: string, orderId: string]} is expanded into \texttt{"arguments.loginId": string, "arguments.orderId": string}.
    \item \textbf{Array of Dict Expansion}: \texttt{results: array[dict[id: number, status: string]]} is expanded into \texttt{"results[].id": array[number], "results[].status": array[string]}.
    \item \textbf{Field Joining}: \texttt{arguments.orderId} can be joined with the elements of \texttt{results} via the \texttt{id} field. The result is a newly joined field: \texttt{"results['joined']": dict[id: number, status: string]}.
\end{itemize}

\paragraph{Clustering.}
The expansion step yields a large set of atomic fields, which are then organized into semantically coherent groups. To manage context length effectively, we cluster fields based on semantic relatedness. Instead of relying on hand-crafted heuristics, we employ an LLM to partition the expanded fields into clusters. For example, identifiers such as \verb|user_id|, \verb|session_id|, and joined fields with matching IDs form one cluster, while numerical values such as \verb|price|, \verb|amount|, and \verb|discount| form another. This LLM-based clustering leverages semantic knowledge to generate meaningful and task-relevant partitions.

Through this pipeline, lengthy and complex logs are transformed into compact, semantically organized structures, enabling lightweight LLMs to effectively generate constraints without exceeding context limitations.

\subsubsection{Constraint Generation}

The constraint generation process of \lighttechname closely resembles that of WebNorm, with the key distinction that its prompts are not manually crafted but automatically derived through the subsequent attack-generation and prompt-refinement loop, making them more suitable for lightweight LLMs. Given the structured logs, \lighttechname first instructs the LLM to produce candidate constraints in the form of executable rules that capture constraints across different fields. These candidates are then iteratively evaluated against training logs, and any violations on normal cases are fed back to the LLM along with contextual information, prompting it to revise or discard the problematic constraints. Through this feedback loop, the system gradually converges to a compact and reliable set of constraints that preserve both structural correctness and semantic consistency.

\subsection{Attack Generation}

Based on the extracted constraints and a pool of normal logs, we deliberately synthesize attack log entries that are difficult for the current constraints to capture. The attack generation process is anchored in the \emph{OWASP API Security Top 10}, one of the most authoritative industry standards for categorizing API vulnerabilities. To align with our log-based setting, we exclude categories that depend primarily on traffic volume or usage frequency (e.g., excessive resource consumption).

The \emph{OWASP API Security Top 10}, maintained by the Open Worldwide Application Security Project (OWASP), serves as the de facto reference for identifying and evaluating API vulnerabilities. It is widely adopted by practitioners, penetration testers, and auditors as a standard checklist for assessing the security of modern web APIs. Its categories are derived from extensive industry data and community feedback, collectively covering the vast majority of real-world API attacks observed in practice.

In our framework, we adopt the OWASP API Security Top 10 as the foundation for guiding attack synthesis. Because our anomaly detection operates at the log level rather than the traffic level, frequency-dependent categories (e.g., rate limiting and resource exhaustion) are excluded. For the remaining categories, we refine them into finer-grained subcategories using LLM-based analysis, ensuring that each synthesized attack corresponds to the log semantics of the target system. Table~\ref{tab:owasptop10} summarizes the OWASP API Security Top 10 categories and indicates their usage in our pipeline.


\begin{table}
    \centering
    \caption{OWASP API Security Top 10 categories and their usage in our framework. Frequency-dependent categories are excluded.}
    \label{tab:owasptop10}
    \myfigfontsize
    \begin{tabular}{p{0.8cm}p{5.2cm}p{6.5cm}}
        \toprule
        \textbf{ID} & \textbf{Category}                               & \textbf{Usage in Our Framework}                                                         \\
        \midrule
        API1        & Broken Object Level Authorization               & Used. generates abnormal logs where access control invariants are bypassed.             \\
        API2        & Broken Authentication                           & Used. simulates login/session anomalies not captured by current invariants.             \\
        API3        & Broken Object Property Level Authorization      & Used. focuses on tampering with specific fields in objects.                             \\
        API4        & Unrestricted Resource Consumption               & Excluded. requires modeling frequency/traffic features.                                 \\
        API5        & Broken Function Level Authorization             & Used. abnormal logs where high-privilege functions are exposed to low-privilege actors. \\
        API6        & Unrestricted Access to Sensitive Business Flows & Used. simulates bypasses of workflow invariants.                                        \\
        API7        & Server-Side Request Forgery                     & Used. generates adversarial logs where external calls are injected.                     \\
        API8        & Security Misconfiguration                       & Used. models cases where abnormal settings or defaults appear in logs.                  \\
        API9        & Improper Inventory Management                   & Excluded. relies on large-scale endpoint enumeration patterns.                          \\
        API10       & Unsafe Consumption of APIs                      & Used. synthesizes abnormal logs involving unvalidated or malicious upstream data.       \\
        \bottomrule
    \end{tabular}
\end{table}

\begin{figure}[t]
    \centering
    \begin{mdframed}[linewidth=0.5pt,roundcorner=0pt,linecolor=black]
        \begin{minipage}{0.95\linewidth}
            \small
            \noindent \textit{** Identify **} You are a security testing expert. To ensure system security, your task is to generate attack logs for a given API or API pair based on the OWASP API Security Top 10 categories.
                {\color{gray}{\textit{[Attack Strategies from OWASP] [Input/Output Format] [Example]}}}

            \noindent \textit{** Input **} {\colorbox{red!20}{\textit{[Normal Log Entries] [Invariant Conditions]}}}

            \noindent \textit{** Output **} {\colorbox{green!20}{\textit{[Abnormal Log Entries]}}}

        \end{minipage}
    \end{mdframed}
    \caption{Abbreviated version of the LLM prompt for attack generation.
        Attack strategies and detailed examples are omitted here for brevity;
        the complete prompt is available in our artifact repository~\cite{agenticnorm-website}.}
    \label{fig:prompt}
\end{figure}


Concretely, for each API and each API pair, we first sample a set of normal log entries. Guided by the OWASP classification and the descriptions of each attack category, we then prompt an LLM to generate corresponding attack log entries. The generated logs are required to bypass the existing constraints whenever possible. These attack entries, together with the sampled normal logs, form a labeled dataset that is subsequently used for prompt refinement. The prompt used for attack generation is shown in Figure~\ref{fig:prompt}, with detailed attack strategies and input/output examples provided in our artifact repository~\cite{agenticnorm-website}.

By grounding attack generation in this taxonomy, our framework inherits both breadth and credibility: it covers a wide spectrum of realistic API threats while remaining fully compatible with our log-based constraints detection setting.


\subsection{Prompt Refinement}


After attack generation, we obtain a labeled dataset consisting of both normal and attack logs. Our next task is to refine the prompts used in constraint generation, so that they can better capture the constraints needed to detect the synthesized attacks. The refinement process is similar to learning a model from labeled data, where the input dataset is the logs and the labels are whether each log is normal or attack. The difference is that instead of adjusting model parameters by policy gradient or backpropagation, we update the prompt text itself using an LLM.

\begin{algorithm}[t]
    \caption{Prompt Refinement via Log-Guided Feedback}
    \label{alg:prompt-refinement}
    \begin{algorithmic}[1]
        \REQUIRE Dataset $D$ containing pairs of normal logs $N$ and abnormal logs $A$
        \REQUIRE Original prompt $P$
        \ENSURE Refined prompt $P'$
        \STATE $M_s \gets [\;]$ \COMMENT{Initialize list of modification suggestions}
        \FOR{each $(N, A) \in D$}
        \STATE $M \gets \text{LLM}(\text{``Generate modification suggestion''}, N, A, P)$
        \COMMENT{Generate modification suggestions based on a normal--abnormal log pair}
        \STATE $M_s.\text{append}(M)$ \COMMENT{Accumulate suggestions across all pairs}
        \ENDFOR
        \STATE $P' \gets \text{LLM}(\text{``Refine prompt''}, P, M_s)$
        \COMMENT{Refine the original prompt by incorporating aggregated suggestions}
        \RETURN $P'$
    \end{algorithmic}
\end{algorithm}



\begin{figure}[t]
    \centering
    % \subcaption[Modification Suggestions]{
    \begin{subfigure}{.96\textwidth}
        \begin{mdframed}[linewidth=0.5pt,roundcorner=0pt,linecolor=black]
            \begin{minipage}{0.95\linewidth}
                \small
                \noindent \textit{** Identify **} You are an expert in prompt engineering and invariant design for API logs. Your role is to iteratively refine prompts so they generate stronger constraints and corresponding Python detection functions. Your should output modification suggestions for the current prompt.
                    {\color{gray}{\textit{[Input/Output Format] [Example]}}}

                \noindent \textit{** Input **} {\colorbox{red!20}{\textit{[Normal Log Entries] [Attack Log Entries] [Current Prompt]}}}

                \noindent \textit{** Output **} {\colorbox{green!20}{\textit{[Modification Suggestions]}}}

            \end{minipage}
        \end{mdframed}
        \caption{Modification Suggestions}
    \end{subfigure}

    \begin{subfigure}{.96\textwidth}
        \begin{mdframed}[linewidth=0.5pt,roundcorner=0pt,linecolor=black]
            \begin{minipage}{0.95\linewidth}
                \small
                \noindent \textit{** Identify **} You are an expert in prompt engineering and invariant design for API logs. Your role is to iteratively refine prompts so they generate stronger constraints and corresponding Python detection functions. You should apply the suggested modifications to the current prompt.
                    {\color{gray}{\textit{[Input/Output Format] [Example]}}}

                \noindent \textit{** Input **} {\colorbox{red!20}{\textit{[Current Prompt] [Modification Suggestions]}}}

                \noindent \textit{** Output **} {\colorbox{green!20}{\textit{[Refined Prompt]}}}

            \end{minipage}
        \end{mdframed}
        \caption{Refined Prompt}
    \end{subfigure}
    \caption{Abbreviated version of the LLM prompt for prompt refinement.
        The complete prompt is available in our artifact repository~\cite{agenticnorm-website}.}
    \label{fig:prompt-refine}
\end{figure}


Algorithm~\ref{alg:prompt-refinement} outlines the prompt refinement process. For each normal-attack log pair in the dataset, we feed it into an LLM along with the current prompt, asking it to generate a modification suggestion. The LLM analyzes the pair and identifies what changes to the prompt could help distinguish between the normal and attack cases. This may involve adding new clauses, modifying existing ones, or removing irrelevant parts. Figure~\ref{fig:prompt-refine} shows an abbreviated version of the prompt used for refinement, with the complete version available in our artifact repository~\cite{agenticnorm-website}.

\subsection{Implementation Details}

\paragraph{LLMs Used.}
\lighttechname is designed to work with lightweight, locally deployable LLMs.
In our study, we observed that the tasks of \emph{Attack Generation} and \emph{Prompt Refinement} place heavier demands on the neural models, as they require more complex reasoning and creative generation. Therefore, we employ larger-scale models for these two tasks, specifically the open-source DeepSeek-V3.
For \emph{Constraint Learning}, the requirements are relatively lower, and we adopt smaller models to balance efficiency and effectiveness. In this work, we experimented with multiple models for constraint learning, including \texttt{gpt-oss-120b}, \texttt{gpt-oss-20b}, \texttt{gemma-3-4b}, and \texttt{DeepSeek-V3}. This hybrid strategy allows us to maintain strong performance while reducing overall system resource consumption and deployment complexity.

\paragraph{Hyperparameters.}
For frequency-based API relation extraction, we set the sliding window size to 20 and select the top-5 most frequent APIs as related APIs.
In field clustering, the maximum expansion depth for nested dictionaries is limited to 3, in order to avoid field explosion from excessive expansion.
For each API, we generate up to 10 normal logs and 10 attack logs for use in prompt refinement.
Prompt refinement is iterated for 10 rounds to ensure that the prompts sufficiently adapt to the synthesized attack scenarios.
Further experimental details can be found in our code repository~\cite{agenticnorm-website}.

\section{Experiments}

We focus on the following research questions.

\begin{itemize}
	\item \textbf{RQ1: Overall Performance.}
	      How effective is \lighttechname in detecting web tamper attacks compared to state-of-the-art baselines and WebNorm?
	      We evaluate its precision, recall, and F1-score on standard benchmarks.

	\item \textbf{RQ2: Ablation Study.}
	      How do the core components of \lighttechname contribute to its performance?
	      We conduct ablation experiments on field clustering, attack generation, and prompt refinement to measure their individual impact.

	\item \textbf{RQ3: Model Scalability.}
	      How does \lighttechname perform when deployed with different scales of lightweight, locally deployable LLMs?
	      We assess the trade-offs between detection accuracy, efficiency, and resource consumption across small, medium, and larger models.

	\item \textbf{RQ4: Direct Substitution.}
	      What happens if WebNorm is directly replaced with a smaller LLM without architectural modifications?
	      This comparison highlights the necessity of our proposed techniques over naïve model substitution.
\end{itemize}

\subsection{Experimental Setup}

\paragraph{Benchmarks.}
We evaluate our approach on two widely-used benchmarks of web application logs:
\trainticket and \nicefish.
Both datasets contain normal and attack traces derived from real-world systems,
with injected tampering behaviors that allow controlled evaluation.
Following prior work, we split logs into fixed-size windows of 20 entries,
and assign binary labels at the window level.

\paragraph{Baselines.}
To demonstrate the effectiveness of \lighttechname,
we compare against three categories of methods:
(1) \textit{learning-based baselines}, including LogRobust~\cite{zhang2019robust},
LogFormer~\cite{guo2024logformer}, and RAPID FastLogAD~\cite{lin2024fastlogad},
which rely on supervised or semi-supervised learning of log sequences;
(2) \textit{rule-based approaches}, represented by WebNorm~\cite{liao2024detecting},
the current state-of-the-art interpretable system for normality modeling.
These baselines cover both predictive and rule-driven paradigms in log anomaly detection.

\paragraph{Evaluation Metrics.}
We adopt precision and recall as the primary metrics.
For windows of normal logs, if any attack is incorrectly flagged, the window is counted as a false positive (FP); otherwise it is a true negative (TN).
For attack-containing windows, the detection of any injected attack is considered a true positive (TP), otherwise it is a false negative (FN).
Formally, precision, recall, and F1-score are computed as
$$\text{Precision} = \dfrac{TP}{TP + FP}, \quad
\text{Recall} = \dfrac{TP}{TP + FN}, \quad
\text{F1} = \dfrac{2 \cdot \text{Precision} \cdot \text{Recall}}{\text{Precision} + \text{Recall}}.$$

\subsection{Results and Analysis}

\begin{table}
    \centering
    \caption{Overall evaluation of \lighttechname}
    \label{tab:overall-eval}
    \begin{tabular}{l|c|c|c|c|c|c}
        \toprule
        \textbf{Model}                    & \multicolumn{3}{c|}{\textbf{TrainTicket}} & \multicolumn{3}{c}{\textbf{NiceFish}}                                                                        \\
        \midrule
                                          & \textbf{Precision}                        & \textbf{Recall}                       & \textbf{F1}   & \textbf{Precision} & \textbf{Recall} & \textbf{F1}   \\
        \midrule
        LogRobust~\cite{zhang2019robust}  & 0.12                                      & 0.65                                  & 0.20          & 0.21               & 0.54            & 0.30          \\
        LogFormer~\cite{guo2024logformer} & 0.27                                      & 0.76                                  & 0.40          & 0.30               & 0.70            & 0.42          \\
        RAPID~\cite{no2023rapid}          & 0.11                                      & 0.90                                  & 0.20          & 0.04               & 1.00            & 0.08          \\
        FastLogAD~\cite{lin2024fastlogad} & 0.04                                      & 0.20                                  & 0.07          & 0.01               & 0.05            & 0.01          \\
        WebNorm~\cite{liao2024detecting}  & 1.00                                      & 0.80                                  & 0.88          & 1.00               & 0.75            & 0.86          \\
        \textbf{\lighttechname (Ours)}    & 1.00                                      & 0.86                                  & \textbf{0.92} & 1.00               & 0.92            & \textbf{0.95} \\
        \bottomrule
    \end{tabular}
\end{table}


\paragraph{RQ1: Overall Performance.}
Table~\ref{tab:overall-eval} summarizes the overall comparison\footnote{We note that our reproduced results of WebNorm differ slightly from those reported in the original paper. After contacting the authors, we confirmed that they updated their dataset, which improved precision but reduced recall. The results shown here reflect this corrected version.}.
\lighttechname achieves the highest F1-scores on both \trainticket (0.92) and \nicefish (0.95).
Some baselines, such as LogFormer and RAPID, obtain relatively high recall
(e.g., 0.90 on \trainticket and 1.00 on \nicefish for RAPID),
but this comes at the cost of extremely low precision (0.11 / 0.04),
leading to many false alarms.
LogFormer offers a more balanced trade-off, but its F1-scores (0.40 / 0.42) remain far lower than ours.
By contrast, WebNorm achieves perfect precision (1.00) but suffers from lower recall
(0.80 on \trainticket and 0.75 on \nicefish),
missing many real attacks due to its reliance on fixed rules.
\lighttechname preserves the perfect precision of WebNorm while substantially improving recall
(0.86 / 0.92), thus delivering the strongest overall detection performance.

\finding{\textbf{RQ1:} \lighttechname surpasses state-of-the-art baselines, achieving the best F1-scores across both benchmarks.}




\begin{table}
    \centering
    \caption{Ablation Study}
    \label{tab:ablation-study}
    \myfigfontsize
    \begin{tabular}{l|c|c}
        \toprule
        \textbf{Model}                     & \textbf{\trainticket} & \textbf{\nicefish} \\
        \midrule
        \textbf{Original (\lighttechname)} & \textbf{0.92}        & \textbf{0.95}     \\
        \midrule
        w/o API relation prediction        & 0.67                 & 0.50              \\
        w/o field clustering               & 0.60                 & 0.83              \\
        w/o prompt refinement              & 0.61                 & 0.83              \\
        \bottomrule
    \end{tabular}
\end{table}

\paragraph{RQ2: Ablation Study.}
Table~\ref{tab:ablation-study} reports the impact of removing each component.
All three modules contribute to performance improvements, but their effects differ in magnitude.
Field clustering proves most critical: removing it reduces the F1-score from 0.92 to 0.60 on \trainticket
and from 0.95 to 0.83 on \nicefish.
Prompt refinement has a comparable impact, with F1 dropping to 0.61 and 0.83, respectively.
By contrast, removing API relation prediction leads to smaller but still notable degradation
(0.92 $\rightarrow$ 0.67 on \trainticket and 0.95 $\rightarrow$ 0.50 on \nicefish),
showing that it provides complementary benefits.


\begin{table}
    \centering
    \caption{Comparison between Number of Tokens with and without Clustering}
    \label{tab:tokens-comparison}
    \begin{tabular}{l|c|c}
        \toprule
                    & \textbf{With Clustering} & \textbf{Without Clustering} \\
        \midrule
        TrainTicket & $7.2 \times 10^3$        & $2.4 \times 10^5$           \\
        NiceFish    & $6.0 \times 10^3$        & $1.5 \times 10^4$           \\
        \bottomrule
    \end{tabular}
\end{table}

To further evaluate the effectiveness of field clustering, we analyze the total number of input tokens in the prompts, comparing settings with and without clustering.
Table~\ref{tab:tokens-comparison} reports the token counts for each invariant generation task. The reduction is particularly pronounced on \trainticket, as its logs contain more fields, and clustering eliminates a larger portion of redundancy. By shortening the token length, the model can process inputs more efficiently, which in turn leads to higher-quality invariants.

\finding{\textbf{RQ2:} Each component of \lighttechname improves performance, with field clustering and prompt refinement being especially crucial.}


\begin{table}
    \centering
    \caption{Comparison of Different LLMs}
    \label{tab:comparing-llms}
    \myfigfontsize
    \begin{tabular}{l|c|c|c|c}
        \toprule
                     & \multicolumn{2}{c|}{\textbf{TrainTicket}} & \multicolumn{2}{c}{\textbf{NiceFish}}                                        \\
        \cmidrule{2-5}
                     & \textbf{Percision}                        & \textbf{Recall}                       & \textbf{Percision} & \textbf{Recall} \\
        \midrule
        DeepSeek-V3  & 1.00                                      & 0.86                                  & 1.00               & 0.95            \\
        Gemma 3 4B   & 1.00                                      & 0.86                                  & 1.00               & 0.95            \\
        GPT-OSS 20B  & 1.00                                      & 0.83                                  & 1.00               & 0.90            \\
        GPT-OSS 120B & 1.00                                      & 0.83                                  & 1.00               & 0.95            \\
        \bottomrule
    \end{tabular}
\end{table}


\paragraph{RQ3: Comparison between Different LLMs.}
Table~\ref{tab:comparing-llms} shows results when varying the LLM used for the
\emph{Constraint Learning} module, while keeping \emph{Attack Generation} and
\emph{Prompt Refinement} fixed to DeepSeek-V3.
Across all four models (\texttt{DeepSeek-V3}, \texttt{Gemma 3 4B}, \texttt{GPT-OSS 20B}, \texttt{GPT-OSS 120B}),
precision remains consistently perfect (1.00),
and recall varies only slightly
(0.83--0.86 on \trainticket and 0.90--0.95 on \nicefish).
This indicates that the effectiveness of \lighttechname is not tied to a specific
model scale in the constraint learning stage.
The invariants derived through clustering and refinement are robust across models,
demonstrating that smaller and more efficient LLMs can be deployed in practice
without sacrificing detection accuracy.

\finding{\textbf{RQ3:} \lighttechname maintains high performance across different LLMs, confirming its adaptability to smaller, locally deployable models.}

\paragraph{RQ4: Direct Substitution.}
To further validate our design, we directly substitute WebNorm’s backbone with a smaller LLM (e.g., \texttt{DeepSeek V3}),
without applying any of our proposed architectural modifications.
Performance drops sharply in recall: on \trainticket, recall falls from 0.92 to 0.60, and on \nicefish, from 0.95 to 0.50.
This experiment shows that naïvely replacing large models with smaller ones is insufficient.
While WebNorm functions well with powerful external LLMs, its invariants are too brittle when scaled down.
By contrast, our techniques—field clustering, attack generation, and prompt refinement—enable small models to remain competitive, supporting practical local deployment.

\finding{\textbf{RQ4:} Simply substituting smaller LLMs into WebNorm leads to severe performance degradation, highlighting the necessity of our architectural innovations for making lightweight deployment viable.}

\section{Discussion}

\textbf{Why can adversarial attacks improve anomaly detection efficiency?}

Adversarial attacks play a crucial role in refining the prompts used for invariant generation. At the initial stage (Round 0), the prompt may fail to capture critical invariants, leading to missed detections for certain types of tamper attacks. However, when we introduce adversarial attacks that exploit these weaknesses, the system is forced to adapt: the failure cases serve as concrete counterexamples that guide the prompt-refinement process. After just one refinement iteration (Round 1), the updated prompt can successfully detect the previously missed anomaly. 

For example, in Round 0, \lighttechname may fail to detect an attack where [\texttt{PLACEHOLDER\_EXAMPLE}], but after one iteration of adversarial attack–guided refinement, the system adjusts the prompt and captures the invariant necessary to identify this anomaly. This self-improving loop demonstrates how adversarial attacks not only test the robustness of the system but also actively drive the enhancement of detection accuracy, ultimately reducing the need for manual intervention and improving efficiency in practice.

\textbf{Rounds of refinement.}


\begin{figure}
    \centering
    \begin{subfigure}{.46\textwidth}
        \centering
        \includegraphics[width=0.8\linewidth]{plot/rounds.pdf}
        \caption{Detection Performance}
        \label{fig:rounds-of-refinement}
    \end{subfigure}
    \quad
    \begin{subfigure}{.46\textwidth}
        \centering
        \includegraphics[width=0.8\linewidth]{plot/number_of_tokens.pdf}
        \caption{Number of Tokens in Prompts}
        \label{fig:number-of-tokens}
    \end{subfigure}
    \caption{Different Rounds of Prompt Refinement}
\end{figure}



\begin{figure}
    \centering
    \includegraphics[width=0.4\linewidth]{plot/number_of_tokens.pdf}
    \caption{Number of Tokens in Prompts in Different Rounds of Refinement}
    \label{fig:number-of-tokens}
\end{figure}


Figure~\ref{fig:rounds-of-refinement} illustrates the impact of iterative prompt refinement on detection performance. We observe that performance improves steadily in the first three rounds, reaching a peak F1-score at Round 3. Beyond this point, performance begins to decline slightly. This trend can be attributed to the increasing length of prompts in later rounds (Figure~\ref{fig:number-of-tokens}), which may lead to overfitting or introduce unnecessary complexity that hampers the model's ability to generalize.
\section{Related Work}

\paragraph{Log Anomaly Detection.}
Early attempts at anomaly detection from logs were largely based on analyzing execution traces of systems or individual APIs, with the goal of spotting deviations from expected behaviors~\cite{ye2024spurious}. 
Classical approaches mostly relied on manually written rules or statistical thresholds~\cite{hansen1993automated,oprea2015detection,prewett2003analyzing,rouillard2004real,roy2015perfaugur,yamanishi2005dynamic,yen2013beehive}. 
While useful in certain domains, these methods require expert-crafted specifications and often fail to generalize across applications.

With the advent of data-driven methods, researchers shifted towards models that automatically learn behavioral patterns from logs~\cite{acharya2007mining,lorenzoli2008automatic,walkinshaw2008inferring,pradel2009automatic,beschastnikh2011leveraging,krka2014automatic,breier2015anomaly,amar2018using,rufino2020improving,stocco2020towards,kang2019spatiotemporal,njoku2025kernel,wu2023effectiveness,lupton2021literature,alam2019framework,schneider2010synoptic}. 
A large body of work in this direction adopts deep learning models to classify log sequences as normal or anomalous. Recurrent architectures have been widely explored to capture sequential dependencies~\cite{du2017deeplog,brown2018recurrent}, while CNN-based solutions exploit local contextual signals~\cite{lu2018detecting,fu2023mlog}. 
More recent efforts employ Transformers~\cite{huang2020hitanomaly,guo2024logformer}, graph neural networks~\cite{zhang2022deeptralog}, or pretrained language models tailored for log data~\cite{guo2021logbert,han2023loggpt}. 
Some studies further reduce labeling overhead by adopting unsupervised or semi-supervised formulations~\cite{yang2021plelog,meng2019loganomaly}. 
Despite their predictive strength, these neural methods often act as black boxes, providing little insight into the root cause of anomalies and occasionally overlooking subtle yet critical deviations.

A smaller set of research emphasizes explainability by constructing explicit normality constraints. The most prominent example is WebNorm~\cite{liao2024detecting}, which encodes behavioral invariants of web systems as first-order logic rules derived from logs. 
This line of work demonstrates that logic-based reasoning can both detect anomalies and provide interpretable explanations. 
However, WebNorm relies heavily on source code analysis and large external models, and it does not explicitly enforce the consistency between logs and their underlying data sources. 
Our work builds on this direction, proposing techniques that refine normality inference with lightweight, deployable models and enhanced relational reasoning.

\paragraph{RESTful API Security.}
RESTful APIs, by virtue of their statelessness and ubiquity, have become an essential surface for attacks in modern web systems. 
Prior research has extensively explored automated vulnerability discovery through API testing and fuzzing~\cite{deng2023nautilus,du2024vulnerability,atlidakis2019restler,viglianisi2020resttestgen,martin2020restest,martin2020automated}. 
These methods typically mutate request sequences or payloads to trigger failures, guided by API specifications, dependency constraints~\cite{viglianisi2020resttestgen,martin2020restest}, or machine learning predictions~\cite{lyu2023miner}. 
Enhanced strategies further refine the fuzzing process to target specific classes of vulnerabilities such as injection attacks or cross-site scripting~\cite{deng2023nautilus,du2024vulnerability}.

While fuzzing uncovers flaws through active probing, our approach takes a complementary perspective: we aim to strengthen API security by learning invariants that characterize normal interaction patterns. 
By detecting deviations from these learned normalities, we provide a systematic way to capture tampering behaviors that bypass conventional fuzzing-based detection.

\paragraph{Comparison to Our Work.}
Existing solutions for log anomaly detection and RESTful API security either emphasize predictive accuracy through deep learning or rely on fuzzing techniques to expose vulnerabilities. 
While effective to some extent, these approaches face notable limitations: deep neural methods lack interpretability and often miss subtle invariants, whereas fuzzing uncovers vulnerabilities opportunistically but does not generalize to unseen tampering strategies. 
WebNorm represents an important step toward explainable anomaly detection, but its reliance on heavyweight, closed-source models and program source code restricts its applicability in practice. 

In contrast, our approach focuses on lightweight, locally deployable models integrated into a multi-agent framework. 
By introducing \emph{field clustering}, we address the long-context challenge inherent in compact models, and by leveraging an iterative \emph{attack-driven prompt refinement loop}, we enable the system to self-improve without extensive manual intervention. 
This combination not only preserves explainability but also ensures that detection can be deployed securely and efficiently in real-world settings.


\section{Conclusion}

In this paper, we presented \lighttechname, a lightweight and locally deployable framework for detecting web tamper attacks from logs. By combining field clustering, adversarial attack generation, and iterative prompt refinement in a multi-agent workflow, \lighttechname addresses key limitations of prior approaches, including dependence on source code, reliance on heavyweight proprietary models, and sensitivity to prompt design. Our evaluation on TrainTicket and NiceFish demonstrates that \lighttechname achieves state-of-the-art performance while remaining robust across different LLM scales.

Looking ahead, we aim to enable \lighttechname to adapt continuously as web applications evolve, allowing it to handle changing APIs and attack patterns with minimal retraining. We also plan to explore transfer and meta-learning techniques so that invariants and refined prompts learned from one system can be effectively reused in new applications, improving both efficiency and generalization across domains.


\bibliographystyle{ACM-Reference-Format}
\bibliography{main}

\end{document}
\endinput
%%
%% End of file `sample-acmsmall-conf.tex'.
