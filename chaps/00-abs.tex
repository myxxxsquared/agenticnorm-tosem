\begin{abstract}
Web applications support important areas such as finance, e-commerce, and healthcare, making their reliability and security of paramount importance. 
However, web frontends are inherently manipulable, and abnormal client behaviors may evade backend checks, creating exploitable vulnerabilities. 
Log analysis has emerged as an effective defense by capturing client-server interactions.
Existing approaches typically learn models from logs, either by training neural network models for anomaly classification or by directly using LLMs for anomaly judgment.
However, these model-learning-based approaches suffer from poor interpretability and high false positive rates, making practical deployment challenging.
To address these limitations, our prior work WebNorm proposed a rule-learning-based approach that extracts explicit constraints from logs, providing stronger interpretability.
However, WebNorm relies heavily on program instrumentation for source code analysis, heavyweight proprietary language models, and manually engineered prompts, limiting its broader applicability.

This paper presents \lighttechname, a significant extension of WebNorm that addresses these limitations through a lightweight anomaly detection framework built upon locally deployable LLMs. 
Building on WebNorm's foundation of constraint-based anomaly detection, we extend the approach with three key innovations: 
(1) eliminating source-code dependence via frequency-based inter-API relation discovery, enabling application to closed-source systems, 
(2) reducing log complexity through field clustering into semantically coherent groups, making lightweight models feasible for large-scale logs, and 
(3) iteratively refining prompts with adversarially generated attack logs, overcoming prompt sensitivity and manual engineering requirements. 
These components are integrated into a novel multi-agent workflow that progressively improves anomaly detection without extensive human intervention.  

We conduct comprehensive experiments on popular benchmarks, including TrainTicket and NiceFish, covering over 220k log entries and 230 attack cases.
Results demonstrate that \lighttechname significantly outperforms the original WebNorm, achieving F1-scores of 0.92 and 0.85 compared to WebNorm's 0.88 and 0.75, while requiring significantly less contextual information and eliminating the need for source code access and proprietary models.
\end{abstract}
