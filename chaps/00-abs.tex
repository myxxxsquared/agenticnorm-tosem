
%%
%% The abstract is a short summary of the work to be presented in the
%% article.
\begin{abstract}
Web applications are increasingly vulnerable to tamper attacks, where subtle manipulations in frontend code lead to abnormal backend logs. Recent work such as WebNorm addresses this challenge by leveraging large proprietary models and source code analysis to generate invariants. However, WebNorm suffers from three limitations: (i) dependency on heavyweight, closed-source models that raise security and deployment concerns, (ii) reliance on program source code, which is often unavailable in practice, and (iii) high sensitivity to prompt engineering, requiring extensive manual intervention. In this paper, we present \lighttechname, a lightweight, locally deployable multi-agent framework for web anomaly detection. To overcome the limited context length of compact LLMs, we propose field clustering, which expands structured logs into semantically grouped fields and generates invariants within each group, effectively reducing input length while preserving critical information. To address prompt sensitivity, we design an attack-generation and prompt-refinement loop, where generated attacks iteratively guide the refinement of prompts to improve invariant quality and detection performance. Together, these components form a self-improving multi-agent workflow that reduces human involvement while achieving performance comparable to WebNorm. Experiments on widely used web application benchmarks demonstrate that \lighttechname achieves detection accuracy and explainability on par with WebNorm, while enabling efficient, secure, and practical deployment.
\end{abstract}