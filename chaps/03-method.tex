\section{Method}


\begin{figure}
    \centering
    \includegraphics[width=.7\textwidth]{figures/method-overview.pdf}
    \caption{Method Overview}
    \label{fig:method_overview}
\end{figure}


In this section, we present \lighttechname, a lightweight multi-agent anomaly detection framework for web applications.
\lighttechname is designed to overcome the limitations of prior solutions such as WebNorm, namely the reliance on heavyweight closed-source models, sensitivity to prompt engineering, and difficulty in handling long log contexts.
Figure~\ref{fig:method_overview} provides an overview of the workflow.
% We first outline the overall workflow, and then describe each component in detail.

\subsection{Workflow}

A central challenge in log-based anomaly detection is that prompt quality strongly influences the detection results. Fixed prompts are brittle and may fail to capture some invariants, leading to missed anomalies. To address this limitation, we propose the an iteration loop, in which attack generation and prompt adjustment are tightly coupled. The loop continuously strengthens prompts by exposing them to adversarial scenarios that exploit their current weaknesses. This iterative consiss of three main modules, forming an iterative loop (Figure~\ref{fig:method_overview}):

\lighttechname consists of three main modules, forming an iterative loop (Figure~\ref{fig:method_overview}):
\begin{itemize}
    \item \textbf{Constraint Learning}: derives invariants from normal logs.
    \item \textbf{Attack Generation}: synthesizes abnormal logs that break or bypass the learned invariants.
    \item \textbf{Prompt Refinement}: updates LLM prompts using feedback from undetected attacks.
\end{itemize}

\lighttechname begins by deriving invariants from the normal logs using an initial prompt by \textbf{Constraint Learning}.
Then, the \textbf{Attack Generation} module synthesizes abnormal logs that break or bypass the learned invariants.
Finally, the \textbf{Prompt Refinement} module updates LLM prompts using feedback from undetected attacks.

% . These invariants encode consistency, semantic, or structural properties of the system. However, they are inherently incomplete and may fail to anticipate novel forms of manipulation. These generated abnormal logs are then combined with the original normal logs, forming a labeled dataset that covers both benign and adversarial scenarios. This dataset becomes the foundation for evaluating and refining the prompts.

% Finally, instead of literal fine-tuning of model parameters, we adopt a \emph{prompt fine-tuning} process. The current invariants are evaluated on the constructed dataset, and each false positive or false negative is fed back into an LLM to solicit a targeted refinement suggestion. The model can propose to add, modify, or delete clauses in the prompt so that the refined prompt is more robust against the observed errors. After collecting refinement suggestions across the dataset, we aggregate them into a revised prompt.

This iterative adversarial loop allows prompts to evolve dynamically. Each cycle expands the attack space by introducing logs that specifically target the weaknesses of the current invariants, and in turn strengthens the prompts by incorporating counterexamples. Over time, this reduces reliance on manual intervention and improves robustness against both known and novel attacks.

Next, we break down each module in detail.

\subsection{Constraint Learning}


\begin{figure}
    \centering
    \includegraphics[width=\textwidth]{figures/method-constraint-learning.pdf}
    \caption{Constraint Learning}
    \label{fig:method_constraint}
\end{figure}


\lighttechname generally follows the idea of WebNorm, but differs in that it does not rely on source code or data-flow analysis.
This requires us to replace several of its original components.
Figure~\ref{fig:method_constraint} illustrates the process of constraint learning.
First, \lighttechname discovers relationships between APIs through frequency-based analysis.
Next, to adapt to lightweight LLMs, \lighttechname applies \emph{Field Clustering}, which reduce the length of the input context per query, thereby lowering the workload of the model while improving its ability to identify constraints.
Finally, \lighttechname adopts a similar approach to WebNorm for detecting both intra-API and inter-API constraints, using an LLM to extract invariants and generate corresponding Python checking code.
Unlike WebNorm, however, the prompts employed here are not manually designed; instead, they are obtained from the iterative refinement process described later, making them better suited for lightweight LLMs.

\subsubsection{Frequency Analysis}

\lighttechname employs a frequency-based method to identify related APIs, eliminating the need for program analysis.
Specifically, for a given API, it scans the surrounding window of log entries and counts the frequency of co-occurring API calls.
The top-$K$ most frequent co-occurrences are considered related APIs, thus establishing inter-API relations.
After this step, we utilize an LLM to verify and filter out spurious relations.

\begin{figure}
    \centering
    \includegraphics[width=\textwidth]{figures/method-freq.pdf}
    \caption{An example of frequency-based analysis.}
    \label{fig:method-freq}
\end{figure}


Figure~\ref{fig:method-freq} shows an example of frequency-based analysis on \trainticket dataset.
Given a list of API calls, we slide a window of size $K$ and count the frequency of co-occurring APIs.
For instance, in this case, \texttt{CreateOrder} and \texttt{AddPassenger} frequently appear together, indicating a potential relationship.
Then LLM is used to verify and filter out spurious relations.

\subsubsection{Field Clustering}

Lightweight LLMs are constrained by limited context windows, making it infeasible to directly process lengthy and complex logs. To address this limitation, we introduce \emph{field clustering}, a technique that decomposes log entries into semantically related groups. This allows invariants to be extracted while ensuring that the input remains within the restricted context length.

To this end, \lighttechname first analyzes the structure of logs, which often contain nested dictionaries and arrays. It then applies a set of expansion rules to flatten these structures into atomic fields. Finally, it employs an LLM to cluster the expanded fields based on semantic relatedness, forming compact groups that can be processed within the context limits.

Figure~\ref{fig:motivating_example_clustering} provides an example of field clustering process. The original log contains nested dicts (e.g., \texttt{arguments}, \texttt{qi}, et al.) and array (\texttt{results}). These structures are first expanded into flat fields (e.g., \texttt{arguments.loginId}, \texttt{arguments.orderId}, et al). Finally, the expanded fields are clustered into semantically related groups (e.g., the cluster of \texttt{arguments.loginId}, \texttt{env.userId},
\texttt{qi.accountId} represents user identifiers).

Formally, we show the field clustering process in three steps: Log Structure Discovery, Expansion, and Clustering.
Log Structure Discovery identifies the schema of logs and their data types.
Expansion applies a set of rules to flatten nested structures into atomic fields.
Clustering groups the expanded fields into semantically related clusters using an LLM.

\paragraph{Log Structure Discovery.}
Each API may produce logs with diverse structures, including nested dictionaries and arrays. We first parse the logs to uncover their structural schema and data types. For each API, \lighttechname analyzes all log entries and infers a unified schema that captures the common structure, represented as fields and their associated types. Formally, we define a recursive grammar for data types:
\[
    \begin{aligned}
        \text{Data} :=\; & \texttt{unknown}                                            \\
        \;|\;            & \texttt{bool}                                               \\
        \;|\;            & \texttt{number}                                             \\
        \;|\;            & \texttt{string}                                             \\
        \;|\;            & \texttt{dict}[key_1:\text{Data}, key_2:\text{Data}, \ldots] \\
        \;|\;            & \texttt{array}[\text{Data}]
    \end{aligned}
\]
Here, \texttt{unknown} denotes cases where the log structure cannot be precisely determined (e.g., due to ambiguity or inconsistency). By aggregating logs across APIs, \lighttechname derives unified schemas that reconcile structural variations.

\paragraph{Expansion.}
After schema discovery, we apply a set of expansion rules to transform nested structures into flat fields. This ensures that all relevant information is explicitly represented, thereby facilitating clustering and invariant extraction.

The rules are as follows:
\begin{itemize}
    \item \textbf{Dict Expansion}: For a dictionary value, each key is concatenated with its parent field using a dot ``.'' separator. Formally, \texttt{d: dict[key: value]} is expanded into \texttt{"d.key": value}.
    \item \textbf{Array of Dict Expansion}: For an array of dictionaries, each dictionary key is expanded to a new array field. Formally, \texttt{a: array[dict[key: value]]} is expanded into \texttt{"a[].key": array[value]}.
    \item \textbf{Field Joining}: In certain cases, meaningful semantics emerge when fields from different structural levels are \emph{joined}. Specifically, if an outer field and an inner field share a common key (e.g., an identifier), we match the entry and promote it as a new joined field.
\end{itemize}

Here are some examples of the expansion rules in motivating example:
\begin{itemize}
    \item \textbf{Dict Expansion}: \texttt{arguments: dict[loginId: string, orderId: string]} is expanded into \texttt{"arguments.loginId": string, "arguments.orderId": string}.
    \item \textbf{Array of Dict Expansion}: \texttt{results: array[dict[id: number, status: string]]} is expanded into \texttt{"results[].id": array[number], "results[].status": array[string]}.
    \item \textbf{Field Joining}: \texttt{arguments.orderId} can be joined with the elements of \texttt{results} via the \texttt{id} field. The result is a new joined field: \texttt{"results['joined']": dict[id: number, status: string]}.
\end{itemize}

\paragraph{Clustering.}
The expansion step yields a large set of atomic fields, which are then organized into semantically coherent groups. To manage context length effectively, we cluster fields based on semantic relatedness. Instead of relying on hand-crafted heuristics, we employ an LLM to partition the expanded fields into clusters. For example, identifiers such as \verb|user_id|, \verb|session_id|, and joined fields with matching IDs form one cluster, while numerical values such as \verb|price|, \verb|amount|, and \verb|discount| form another. This LLM-based clustering leverages semantic knowledge to generate meaningful and task-relevant partitions.

Through this pipeline, lengthy and complex logs are transformed into compact, semantically organized structures, enabling lightweight LLMs to effectively generate invariants without exceeding context limitations.

\subsubsection{Invariant Generation}

The invariant generation process of \lighttechname closely resembles that of WebNorm, with the key distinction that its prompts are not manually crafted but automatically derived through the subsequent attack-generation and prompt-refinement loop, making them more suitable for lightweight LLMs. Given the structured logs, \lighttechname first instructs the LLM to produce candidate invariants in the form of executable rules that capture constraints across different fields. These candidates are then iteratively evaluated against training logs, and any violations on normal cases are fed back to the LLM along with contextual information, prompting it to revise or discard the problematic invariants. Through this feedback loop, the system gradually converges to a compact and reliable set of invariants that preserve both structural correctness and semantic consistency.

\subsection{Attack Generation}

Based on the extracted invariants and a pool of normal logs, we deliberately generate abnormal logs that are difficult for the current invariants to detect. The attack generation process is guided by the \emph{OWASP API Security Top 10}, which is one of the most authoritative industry standards for summarizing API vulnerabilities. To make the framework compatible with our log-based setting, we exclude frequency- and usage-based attacks (e.g., excessive resource consumption).

To ground our attack generation process in widely recognized security standards, we adopt the \emph{OWASP API Security Top 10}. This framework, maintained by the Open Worldwide Application Security Project (OWASP), is one of the most authoritative industry references for API vulnerabilities. It is widely used by practitioners, penetration testers, and auditors as the de facto checklist for assessing the security of modern web APIs. The categories are derived from extensive industry data and community input, and they collectively cover the vast majority of real-world API attacks reported across the web.

In our setting, we take the OWASP API Top 10 as a foundation for guiding attack synthesis. Since our log-based anomaly detection framework does not model traffic-level features, we exclude frequency-dependent categories (e.g., rate limiting issues, excessive resource consumption). For the remaining categories, we further refine them into subcategories using LLM-based analysis, ensuring that each attack is tailored to the log semantics of the system under study.


\begin{table}
    \centering
    \caption{OWASP API Security Top 10 categories and their usage in our framework. Frequency-dependent categories are excluded.}
    \label{tab:owasptop10}
    \myfigfontsize
    \begin{tabular}{p{0.8cm}p{5.2cm}p{6.5cm}}
        \toprule
        \textbf{ID} & \textbf{Category}                               & \textbf{Usage in Our Framework}                                                         \\
        \midrule
        API1        & Broken Object Level Authorization               & Used. generates abnormal logs where access control invariants are bypassed.             \\
        API2        & Broken Authentication                           & Used. simulates login/session anomalies not captured by current invariants.             \\
        API3        & Broken Object Property Level Authorization      & Used. focuses on tampering with specific fields in objects.                             \\
        API4        & Unrestricted Resource Consumption               & Excluded. requires modeling frequency/traffic features.                                 \\
        API5        & Broken Function Level Authorization             & Used. abnormal logs where high-privilege functions are exposed to low-privilege actors. \\
        API6        & Unrestricted Access to Sensitive Business Flows & Used. simulates bypasses of workflow invariants.                                        \\
        API7        & Server-Side Request Forgery                     & Used. generates adversarial logs where external calls are injected.                     \\
        API8        & Security Misconfiguration                       & Used. models cases where abnormal settings or defaults appear in logs.                  \\
        API9        & Improper Inventory Management                   & Excluded. relies on large-scale endpoint enumeration patterns.                          \\
        API10       & Unsafe Consumption of APIs                      & Used. synthesizes abnormal logs involving unvalidated or malicious upstream data.       \\
        \bottomrule
    \end{tabular}
\end{table}

Table~\ref{tab:owasptop10} summarizes the OWASP API Security Top 10 categories and indicates their usage in our pipeline.

By leveraging this taxonomy, our attack generation process inherits both breadth and credibility: it covers a wide spectrum of realistic API threats while remaining compatible with our log-based invariant detection setting.

\subsection{Prompt Refinement}

After attack generation, we obtain a labeled dataset consisting of both normal and abnormal logs. Our next task is to refine the prompts used in invariant generation, so that they can better capture the constraints needed to detect the synthesized attacks. The refinement process is similar to learning a model from labeled data, where the input dataset is the logs and the labels are whether each log is normal or abnormal. The difference is that instead of adjusting model parameters by policy gradient or backpropagation, we update the prompt text itself using an LLM.

\begin{algorithm}[t]
    \caption{Prompt Refinement via Log-Guided Feedback}
    \label{alg:prompt-refinement}
    \begin{algorithmic}[1]
        \REQUIRE Dataset $D$ containing pairs of normal logs $N$ and abnormal logs $A$
        \REQUIRE Original prompt $P$
        \ENSURE Refined prompt $P'$
        \STATE $M_s \gets [\;]$ \COMMENT{Initialize list of modification suggestions}
        \FOR{each $(N, A) \in D$}
        \STATE $M \gets \text{LLM}(\text{``Generate modification suggestion''}, N, A, P)$
        \COMMENT{Generate modification suggestions based on a normal--abnormal log pair}
        \STATE $M_s.\text{append}(M)$ \COMMENT{Accumulate suggestions across all pairs}
        \ENDFOR
        \STATE $P' \gets \text{LLM}(\text{``Refine prompt''}, P, M_s)$
        \COMMENT{Refine the original prompt by incorporating aggregated suggestions}
        \RETURN $P'$
    \end{algorithmic}
\end{algorithm}


Algorithm~\ref{alg:prompt-refinement} outlines the prompt refinement process. For each normal-abnormal log pair in the dataset, we feed it into an LLM along with the current prompt, asking it to generate a modification suggestion. The LLM analyzes the pair and identifies what changes to the prompt could help distinguish between the normal and abnormal cases. This may involve adding new clauses, modifying existing ones, or removing irrelevant parts.
