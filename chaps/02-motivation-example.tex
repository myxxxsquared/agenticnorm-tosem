\section{Motivating Example}

\begin{figure}
    \centering
    \includegraphics[width=\textwidth]{figures/motivating-example.pdf}
    \caption{A motivating example from the \trainticket dataset illustrating a ticket refund workflow, 
where the user first retrieves a list of orders using \texttt{queryOrders} and then requests a refund through \texttt{cancelOrder}. 
The blue highlights indicate identifiers that should match across the two APIs, while the red highlight marks an attack case, 
where the \texttt{orderId} is replaced with a value not present in the queried list.}
    \label{fig:motivating_example}
\end{figure}


To illustrate the challenges of detecting anomalies from API logs, we examine a case from the \trainticket dataset~\cite{trainticketsystem} involving a compromised ticket refund process. 
We focus on two backend APIs that appear in the logs: 
\texttt{/api/v1/queryOrders} (\texttt{queryOrders}) and \texttt{/api/v1/cancelOrder} (\texttt{cancelOrder}). 
Figure~\ref{fig:motivating_example} shows a simplified version of the relevant log entries, while the complete logs are provided in our anonymous artifact~\cite{agenticnorm-website}. The original logs are lengthy and represented in JSON format.  

In a normal cancellation workflow, the operation consists of two steps. 
First, the user invokes \texttt{queryOrders} to retrieve a list of refundable tickets. 
Second, the user selects one of the returned tickets and issues a cancellation request via \texttt{cancelOrder}. 
In this case, the \texttt{orderId} used for cancellation (e.g., \texttt{fe9c72d9}) must be one of the identifiers returned by \texttt{queryOrders}.  

However, because frontend code executes entirely on the client side, a malicious user can tamper with browser data and craft an unauthorized request. 
For instance, the attacker may replace the legitimate \texttt{orderId} with an arbitrary identifier not included in the queried list (e.g., \texttt{418ea03c}). 
This manipulation allows the attacker to cancel a ticket they do not own or repeat a cancellation that should be disallowed. 
Such behavior may lead to duplicate refunds and financial losses, creating serious risks for system security and integrity.  


\subsection{Background}
Here, we briefly introduce the concepts of APIs and logs in web applications, and illustrate them with the example in Figure~\ref{fig:motivating_example}.  
An API serves as the communication interface between the frontend and backend, typically specified by a request path and a payload format. 
 In this example, there are two APIs. The first API, \texttt{queryOrders}, returns a list of refundable tickets, each with a unique \texttt{orderId}. The second API, \texttt{cancelOrder}, takes an \texttt{orderId} as input to process a cancellation request.

A log records the actual data exchanged through an API call, usually including the request path and a JSON-structured object for both request and response.
Thus, logs are often stored in structured JSON format.  
In the given example, for each API, we show its corresponding log entry in a simplified format.

For anomaly detection, we consider two types of logs: normal and abnormal.
Logs generated by normal browser operations are considered benign, while those modified or replayed on the client side are treated as abnormal.

In a normal case, the second API, \texttt{cancelOrder}, should use one of these \texttt{orderId}s when submitting a cancellation request (e.g., \texttt{fe9c72d9}), this invariant is enforced by the frontend code.
However, an attacker may tamper with the frontend and replace this value with an arbitrary \texttt{orderId} not included in the list (e.g., \texttt{418ea03c}), thereby cancelling a ticket they do not own.  

The relation that \texttt{cancelOrder.arguments.orderId} must match one of \texttt{queryOrders.results[].id} is an example of a cross-API dependency. 
WebNorm refers to such relations as \emph{dependencies} and identifies them through program analysis, source code instrumentation, or other static/dynamic methods. 
Once the dependency is known, any violation of it in the logs can be flagged as abnormal behavior.

\subsection{Existing Approaches}  
Existing log-based anomaly detection methods can be divided into two categories:  
(1) \textbf{Model-learning-based detectors}, which learn embeddings or features from normal and abnormal logs and classify anomalies based on learned patterns; and  
(2) \textbf{Invariant-learning-based detectors}, which derive semantic invariants from logs and flag violations as anomalies.  

Model-learning-based detectors face difficulties in this scenario. Normal and abnormal logs differ in only a few fields, making them nearly indistinguishable in embedding space. 
Moreover, the anomaly signal is diluted by long sequences interleaved with irrelevant events, further reducing detection accuracy.  

To address these issues, WebNorm~\cite{liao2024detecting} learns semantic invariants from logs using LLMs. 
For example, it can capture the constraint that \texttt{cancelOrder.arguments.orderId} must match one of \texttt{queryOrders.results[].id}. 
WebNorm maps code-level data flows to log fields and asks an LLM to validate them as invariants. At runtime, any violation of such constraints is flagged as an anomaly.  
In this motivating case, WebNorm successfully identifies the cross-API constraint and detects the abnormal behavior.  

Despite its effectiveness, WebNorm has three key limitations:  
\begin{itemize}
    \item \textbf{Dependence on program analysis and source code}: it requires access to and instrumentation of frontend/backend code to align logs with program workflows, which is costly, fragile under rapid iteration, and infeasible for closed-source or third-party systems.  
    \item \textbf{Reliance on heavyweight proprietary LLMs}: invariant synthesis depends on closed-source remote models, leading to latency, cost, and compliance/privacy concerns. Real logs are long and deeply nested, often exceeding the context window of compact models.  
    \item \textbf{Prompt sensitivity}: correct invariants often emerge only when carefully engineered prompts are used, making the process labor-intensive, project-specific, and difficult to generalize.  
\end{itemize}  

\subsection{Our Approach}  
We aim to preserve WebNorm’s strength in capturing \emph{consistency constraints} while addressing its limitations. 
Our approach uses only logs (no program analysis), relies primarily on compact, locally deployable LLMs, and mitigates prompt fragility by automatically refining prompts with generated abnormal logs.  

Specifically, we propose two techniques for anomaly detection with compact models:  
(1) \emph{Field Clustering}: log entities are expanded into flattened fields and grouped into small, semantically coherent clusters. Each cluster is processed separately by the LLM, reducing context length and highlighting field-level relationships; and  
(2) \emph{Prompt Refinement via Generated Attacks}: prompts are designed to guide the LLM to generate abnormal logs. These logs reveal missing constraints, which are then used to refine the prompts. The refined prompts enable compact models to generate more accurate invariants.  

\paragraph{Field Clustering}  
To handle compact models efficiently, we expand each JSON log into flattened fields and group comparable fields into clusters. 
Figure~\ref{fig:motivating_example_clustering} illustrates this process. Each cluster is provided as a separate input to the LLM, ensuring that semantically related fields are explicitly compared.  

\begin{figure}
    \centering
    \includegraphics[width=\textwidth]{figures/motivating-example-clustering.pdf}
    \caption{An illustration of \emph{Field Clustering} on the motivating example from the \trainticket dataset. The original log entity is decomposed into flatten fields and then clustered into clusters, each containing only related and comparable fields.}
    \label{fig:motivating_example_clustering}
\end{figure}
  

\paragraph{Prompt Refinement via Generated Attacks}  

\begin{figure}
    \centering
    \includegraphics[width=\textwidth]{figures/motivating-example-prompt.pdf}
    \caption{An illustration of different prompts. The left panel shows the original prompt used in WebNorm, which relies on manually crafted instructions and examples, making prompt adjustment highly labor-intensive. The middle panel adds new instructions to the original prompt, guiding the LLM to generate abnormal logs. The right panel presents the refined prompt, which is able to capture a broader set of invariants.}
    \label{fig:motivating_example_prompt}
\end{figure}
  

Compact models often lack reasoning power and cannot reliably infer constraints from fixed prompts. 
Manually adjusting prompts is both labor-intensive and project-specific. 
Figure~\ref{fig:motivating_example_prompt} (left) shows the original WebNorm prompt, which depends on handcrafted instructions. 
Our approach (middle) enhances the prompt by adding instructions that guide the model to generate abnormal logs. These generated logs help the model reason about field relationships—for example, ensuring consistency between joined fields and their originals. 
As shown in Figure~\ref{fig:motivating_example_prompt} (right), the refined prompt enables the model to produce invariants that successfully capture the intended field-level constraints.
