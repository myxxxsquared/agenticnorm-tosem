\section{Discussion}


\begin{figure}
    \centering
    \includegraphics[width=\textwidth]{figures/example.pdf}
    \caption{
        An example from \trainticket illustrating how prompt refinement improves the quality of generated constraints. 
        The left panel shows a simplified version of the log entity, the middle panel presents the refined prompt with the additional range constraints compared to the original prompt, and the right panel displays the generated invariant. 
    }
    \label{fig:example}
\end{figure}

\subsection{Impact of Adversarial Attacks on Prompt Refinement}

Adversarial attacks play a crucial role in refining the prompts used for constraint generation. At the initial stage (Round 0), the prompt may fail to capture critical constraints, leading to missed detections for certain types of tamper attacks. However, when we introduce adversarial attacks that exploit these weaknesses, the system is forced to adapt: the failure cases serve as concrete counterexamples that guide the prompt-refinement process. After just one refinement iteration (Round 1), the updated prompt can successfully detect the previously missed anomaly.

For example, as shown in Figure~\ref{fig:example}, in Round 0, \lighttechname fails to detect an attack where \texttt{arguments.oti.assurance} is assigned a negative value (\texttt{-2}). In the first iteration of attack generation, \lighttechname produces an attack in the \textit{Unsafe Consumption of APIs} category, which involves assigning out-of-range values to fields. The refined prompt introduces explicit range constraints on numeric fields, resulting in the learned constraint. This new constraint allows the system to correctly identify the anomaly that was previously missed.

This self-improving loop demonstrates how adversarial attacks not only test the robustness of the system but also actively drive the enhancement of detection accuracy, ultimately reducing the need for manual intervention and improving efficiency in practice.


\subsection{Effect of Iterative Prompt Refinement}


\begin{figure}
    \centering
    \begin{subfigure}{.46\textwidth}
        \centering
        \includegraphics[width=0.8\linewidth]{plot/rounds.pdf}
        \caption{Detection Performance}
        \label{fig:rounds-of-refinement}
    \end{subfigure}
    \quad
    \begin{subfigure}{.46\textwidth}
        \centering
        \includegraphics[width=0.8\linewidth]{plot/number_of_tokens.pdf}
        \caption{Number of Tokens in Prompts}
        \label{fig:number-of-tokens}
    \end{subfigure}
    \caption{Different Rounds of Prompt Refinement}
\end{figure}



\begin{figure}
    \centering
    \includegraphics[width=0.4\linewidth]{plot/number_of_tokens.pdf}
    \caption{Number of Tokens in Prompts in Different Rounds of Refinement}
    \label{fig:number-of-tokens}
\end{figure}


Figure~\ref{fig:rounds-of-refinement} presents the effect of iterative prompt refinement on detection performance across different models. We observe that recall improves steadily during the early rounds, as each iteration introduces additional constraints that the models can leverage for anomaly detection. Most models reach their highest performance by round 5-6, after which additional refinements yield diminishing returns and performance stabilizes. This pattern suggests that iterative refinement is highly effective in the initial phase, where previously missing constraints are incrementally uncovered, but exhibits saturation once the critical set of constraints has been captured.

This trend aligns with Figure~\ref{fig:number-of-tokens}, which shows that the number of tokens in the refined prompts continues to increase with each round. While longer prompts provide more detailed constraints, they eventually add little marginal benefit, indicating that the models have already captured the critical constraints needed for detection. Beyond this stage, further refinement mainly increases prompt complexity without improving effectiveness.

% This observation is further supported by Figure~\ref{fig:number-of-tokens}, which shows that the number of tokens in the refined prompts continues to grow with each round. While longer prompts provide richer descriptions and more detailed constraints, the marginal benefit decreases over time. Once the essential semantic rules are incorporated, further prompt expansion mainly increases textual complexity without translating into measurable performance gains. In practice, this indicates a natural stopping point for refinement: beyond 5-6 rounds, additional iterations introduce overhead for both model inference and human interpretability, but do not substantially improve detection quality.

\subsection{False Negatives}


\begin{figure}
    \centering
    \includegraphics[width=.5\textwidth]{figures/example-fn.pdf}
    \caption{
        An example from \trainticket that cannot be detected by \lighttechname. In this example, the range of a given field should be 0, 1, 2, 3 according to the system documentation. However, \lighttechname do not know this information and thus cannot detect the tampering when the field is set to 5.
    }
    \label{fig:example-fn}
\end{figure}

Although \lighttechname achieves consistently high precision and recall, some false negatives inevitably remain. Figure~\ref{fig:example-fn} illustrates one such case from \trainticket where \lighttechname fails to detect the anomaly. In this example, the field \texttt{arguments.oti.assurance} is documented to only take values in the range \{0, 1, 2, 3\}. When the field is tampered with and set to an out-of-range value (e.g., 5), the violation should be flagged as anomalous. However, because \lighttechname does not have direct access to external domain knowledge such as system documentation or API specifications, it cannot infer that the injected value is invalid. 
This limitation highlights an important trade-off in our design: the reliance on constraints inferred purely from runtime logs ensures broad applicability in scenarios where source code or documentation is unavailable, but it also means that certain domain-specific invariants may not be captured.


\subsection{Limitations}
Despite the promising results of \lighttechname, several limitations remain.

\paragraph{Benchmark Selection.}
Our evaluation is restricted to two benchmarks (TrainTicket and NiceFish).
While these datasets are widely used in prior work, they may not fully reflect the diversity of real-world applications, especially in large-scale industrial or domain-specific systems.
Moreover, our attack synthesis is guided by the OWASP API Security Top 10, which provides strong coverage of common vulnerabilities but may overlook emerging or highly specialized attack patterns.
This suggests that future work should consider integrating adaptive or domain-specific attack taxonomies.

\paragraph{Model Dependencies.}
Although \lighttechname is designed for lightweight deployment, certain modules (e.g., attack generation and prompt refinement) still rely on larger open-source models.
This hybrid strategy ensures effectiveness but may limit applicability in environments with strict computational or resource constraints.
In addition, the iterative prompt refinement process increases prompt length and complexity, which may reduce efficiency and introduce difficulties in maintaining refined prompts at scale.
